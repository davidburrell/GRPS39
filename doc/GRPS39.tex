% generated by GAPDoc2LaTeX from XML source (Frank Luebeck)
\documentclass[a4paper,11pt]{report}

            \usepackage{a4wide}
            \newcommand{\bbZ}{\mathbb{Z}}
        
\usepackage[top=37mm,bottom=37mm,left=27mm,right=27mm]{geometry}
\sloppy
\pagestyle{myheadings}
\usepackage{amssymb}
\usepackage[utf8]{inputenc}
\usepackage{makeidx}
\makeindex
\usepackage{color}
\definecolor{FireBrick}{rgb}{0.5812,0.0074,0.0083}
\definecolor{RoyalBlue}{rgb}{0.0236,0.0894,0.6179}
\definecolor{RoyalGreen}{rgb}{0.0236,0.6179,0.0894}
\definecolor{RoyalRed}{rgb}{0.6179,0.0236,0.0894}
\definecolor{LightBlue}{rgb}{0.8544,0.9511,1.0000}
\definecolor{Black}{rgb}{0.0,0.0,0.0}

\definecolor{linkColor}{rgb}{0.0,0.0,0.554}
\definecolor{citeColor}{rgb}{0.0,0.0,0.554}
\definecolor{fileColor}{rgb}{0.0,0.0,0.554}
\definecolor{urlColor}{rgb}{0.0,0.0,0.554}
\definecolor{promptColor}{rgb}{0.0,0.0,0.589}
\definecolor{brkpromptColor}{rgb}{0.589,0.0,0.0}
\definecolor{gapinputColor}{rgb}{0.589,0.0,0.0}
\definecolor{gapoutputColor}{rgb}{0.0,0.0,0.0}

%%  for a long time these were red and blue by default,
%%  now black, but keep variables to overwrite
\definecolor{FuncColor}{rgb}{0.0,0.0,0.0}
%% strange name because of pdflatex bug:
\definecolor{Chapter }{rgb}{0.0,0.0,0.0}
\definecolor{DarkOlive}{rgb}{0.1047,0.2412,0.0064}


\usepackage{fancyvrb}

\usepackage{mathptmx,helvet}
\usepackage[T1]{fontenc}
\usepackage{textcomp}


\usepackage[
            pdftex=true,
            bookmarks=true,        
            a4paper=true,
            pdftitle={Written with GAPDoc},
            pdfcreator={LaTeX with hyperref package / GAPDoc},
            colorlinks=true,
            backref=page,
            breaklinks=true,
            linkcolor=linkColor,
            citecolor=citeColor,
            filecolor=fileColor,
            urlcolor=urlColor,
            pdfpagemode={UseNone}, 
           ]{hyperref}

\newcommand{\maintitlesize}{\fontsize{50}{55}\selectfont}

% write page numbers to a .pnr log file for online help
\newwrite\pagenrlog
\immediate\openout\pagenrlog =\jobname.pnr
\immediate\write\pagenrlog{PAGENRS := [}
\newcommand{\logpage}[1]{\protect\write\pagenrlog{#1, \thepage,}}
%% were never documented, give conflicts with some additional packages

\newcommand{\GAP}{\textsf{GAP}}

%% nicer description environments, allows long labels
\usepackage{enumitem}
\setdescription{style=nextline}

%% depth of toc
\setcounter{tocdepth}{1}





%% command for ColorPrompt style examples
\newcommand{\gapprompt}[1]{\color{promptColor}{\bfseries #1}}
\newcommand{\gapbrkprompt}[1]{\color{brkpromptColor}{\bfseries #1}}
\newcommand{\gapinput}[1]{\color{gapinputColor}{#1}}


\begin{document}

\logpage{[ 0, 0, 0 ]}
\begin{titlepage}
\mbox{}\vfill

\begin{center}{\maintitlesize \textbf{ GRPS39 \mbox{}}}\\
\vfill

\hypersetup{pdftitle= GRPS39 }
\markright{\scriptsize \mbox{}\hfill  GRPS39  \hfill\mbox{}}
{\Huge \textbf{ Library of the groups of order 19683 \mbox{}}}\\
\vfill

{\Huge  0.1 \mbox{}}\\[1cm]
{ 31 July 2022 \mbox{}}\\[1cm]
\mbox{}\\[2cm]
{\Large \textbf{ David Burrell\\
   \mbox{}}}\\
\hypersetup{pdfauthor= David Burrell\\
   }
\end{center}\vfill

\mbox{}\\
{\mbox{}\\
\small \noindent \textbf{ David Burrell\\
   }  Email: \href{mailto://davidburrell@ufl.edu} {\texttt{davidburrell@ufl.edu}}\\
  Homepage: \href{https://davidburrell.github.io/} {\texttt{https://davidburrell.github.io/}}}\\
\end{titlepage}

\newpage\setcounter{page}{2}
\newpage

\def\contentsname{Contents\logpage{[ 0, 0, 1 ]}}

\tableofcontents
\newpage

     
\chapter{\textcolor{Chapter }{Groups of Order 19683}}\label{Chapter_Groups_of_Order_19683}
\logpage{[ 1, 0, 0 ]}
\hyperdef{L}{X79A8D55C7818F30C}{}
{
  
\section{\textcolor{Chapter }{Overview}}\label{Chapter_Groups_of_Order_19683_Section_Overview}
\logpage{[ 1, 1, 0 ]}
\hyperdef{L}{X8389AD927B74BA4A}{}
{
  

 This library gives complete access to the following groups of order 19683: 
\begin{itemize}
\item  The rank 1 group 
\item  All rank 2 groups 
\item  All rank 3 groups with p-class not equal to 3 
\item  All rank 4 groups with p-class at least 4 
\item  All rank 5 groups with p-class at least 4 
\item  All rank 6 groups with p-class at least 4 
\item  All rank 7 groups with p-class at least 4 
\item  All rank 8 groups with p-class at least 3 
\item  The rank 9 group 
\end{itemize}
 

 This library gives partial information on the remaining groups of order 19683: 

 
\begin{itemize}
\item  Rank 3 groups with p-class 3 
\item  Rank 4 groups with p-class 2 
\item  Rank 4 groups with p-class 3 
\item  Rank 5 groups with p-class 2 
\item  Rank 5 groups with p-class 3 
\item  Rank 6 groups with p-class 2 
\item  Rank 6 groups with p-class 3 
\item  Rank 7 groups with p-class 2 
\item  Rank 7 groups with p-class 3 
\end{itemize}
 

 For the groups that are not explicity available the following information is
available: 

 
\begin{itemize}
\item  Parent Group ID 
\item  Parent Group Order 
\item  p-class 
\item  Rank 
\item  Age 
\end{itemize}
 

 The groups are sorted first by their parent group ids and then by the pc codes
of the standard presentations for the groups. The data contained in this
library was used in the 2022 enumeration of the groups of order 19683 \cite{Burrell2022a}. The computational tools used were developed in the 2021 enumeration of the
groups of order 1024 \cite{Burrell2021a}. The available groups were generated using the p-group generation algorithm \cite{OBrien1990a} as implemented in the ANUPQ package \cite{Gamble2019a}. The information on the remaining groups was calculated using the
cohomological methods for enumerating p-groups as introduced in \cite{Eick1999a}. 

 }

 }

   
\chapter{\textcolor{Chapter }{Functionality}}\label{Chapter_Functionality}
\logpage{[ 2, 0, 0 ]}
\hyperdef{L}{X87F1120883F5B4D0}{}
{
  

 
\section{\textcolor{Chapter }{Methods}}\label{Chapter_Functionality_Section_Methods}
\logpage{[ 2, 1, 0 ]}
\hyperdef{L}{X8606FDCE878850EF}{}
{
  

 Once the package is loaded the user may call \texttt{SmallGroup(1024,i)} and receive either a group if available or a \emph{partially constructed group} which has the following attributes set 
\begin{itemize}
\item  p-class 
\item  Rank 
\item  Heritage 
\item  Order 
\end{itemize}
 

 
\begin{Verbatim}[commandchars=!@|,fontsize=\small,frame=single,label=Example]
  !gapprompt@gap>| !gapinput@SmallGroup(19683,1);|
  <pc group of size 19683 with 9 generators>
  !gapprompt@gap>| !gapinput@G:=SmallGroup(19683,1);|
  <pc group of size 19683 with 9 generators>
  !gapprompt@gap>| !gapinput@RankPGroup(G);|
  3
  !gapprompt@gap>| !gapinput@PClassPGroup(G);|
  2
  !gapprompt@gap>| !gapinput@GRPS39_Heritage(G);|
  [ 27, 5, 1 ]
  !gapprompt@gap>| !gapinput@H:=SmallGroup(19683,2); #this is a partially constructed group|
  <pc group with 0 generators>
  !gapprompt@gap>| !gapinput@PClassPGroup(H);|
  2
  !gapprompt@gap>| !gapinput@RankPGroup(H);|
  4
  !gapprompt@gap>| !gapinput@GRPS39_Heritage(H);|
  [ 81, 15, 1 ]
  !gapprompt@gap>| !gapinput@K:=SmallGroup(19683,3); #this is a partially constructed group|
  <pc group with 0 generators>
  !gapprompt@gap>| !gapinput@PClassPGroup(K);|
  2
  !gapprompt@gap>| !gapinput@RankPGroup(K);|
  4
  !gapprompt@gap>| !gapinput@GRPS39_Heritage(K);|
  [ 81, 15, 2 ]
  #notice that H,K have the same parent group but their age differs
\end{Verbatim}
 

\subsection{\textcolor{Chapter }{GRPS39{\textunderscore}AvailableMap}}
\logpage{[ 2, 1, 1 ]}\nobreak
\hyperdef{L}{X7F11E5C887F981D6}{}
{\noindent\textcolor{FuncColor}{$\triangleright$\enspace\texttt{GRPS39{\textunderscore}AvailableMap({\mdseries\slshape N})\index{GRPS39{\textunderscore}AvailableMap@\texttt{GRPS39{\textunderscore}AvailableMap}}
\label{GRPS39uScoreAvailableMap}
}\hfill{\scriptsize (function)}}\\
\textbf{\indent Returns:\ }
\texttt{int} 



 For $1 \leq i \leq 203,045,160$ this function will return the SmallGroup ID of the $i$th available group among all the groups of order 19683. }

 
\begin{Verbatim}[commandchars=!@|,fontsize=\small,frame=single,label=Example]
  #group 1 is available
  #groups 2-66668 are not available
  #SmallGroup(19683,66668) is not available
  !gapprompt@gap>| !gapinput@g:=SmallGroup(19683,66668);|
  <pc group with 0 generators> #this is a partially constructed group
  !gapprompt@gap>| !gapinput@g:=SmallGroup(19683,66669);|
  <pc group of size 19683 with 9 generators> #this is an available group
  !gapprompt@gap>| !gapinput@GRPS39_AvailableMap(2);|
  66669
  #access the ith available group of order 19683
  !gapprompt@gap>| !gapinput@SmallGroup(19683,GRPS39_AvailableMap(i)); #for i <= 203,045,160|
\end{Verbatim}
 

\subsection{\textcolor{Chapter }{GRPS39{\textunderscore}InverseAvailableMap}}
\logpage{[ 2, 1, 2 ]}\nobreak
\hyperdef{L}{X8304B1327C1570E0}{}
{\noindent\textcolor{FuncColor}{$\triangleright$\enspace\texttt{GRPS39{\textunderscore}InverseAvailableMap({\mdseries\slshape N})\index{GRPS39{\textunderscore}InverseAvailableMap@\texttt{GRP}\-\texttt{S39{\textunderscore}}\-\texttt{Inverse}\-\texttt{AvailableMap}}
\label{GRPS39uScoreInverseAvailableMap}
}\hfill{\scriptsize (function)}}\\
\textbf{\indent Returns:\ }
\texttt{int} 



 For $1 \leq i \leq 5,937,876,645$ if \texttt{SmallGroup(19683,i)} is available this will return its position in the available groups list or
else it will print a message telling you that it is not available and return
0. }

 
\begin{Verbatim}[commandchars=!@|,fontsize=\small,frame=single,label=Example]
  !gapprompt@gap>| !gapinput@GRPS39_InverseAvailableMap(GRPS39_AvailableMap(i)) = i;|
  !gapprompt@gap>| !gapinput@GRPS39_InverseAvailableMap(2);|
  This is an immediate descendant of 81#15 and is not available
  0
\end{Verbatim}
 

\subsection{\textcolor{Chapter }{GRPS39{\textunderscore}Heritage (for IsGroup)}}
\logpage{[ 2, 1, 3 ]}\nobreak
\hyperdef{L}{X85731F0B870514FD}{}
{\noindent\textcolor{FuncColor}{$\triangleright$\enspace\texttt{GRPS39{\textunderscore}Heritage({\mdseries\slshape G})\index{GRPS39{\textunderscore}Heritage@\texttt{GRPS39{\textunderscore}Heritage}!for IsGroup}
\label{GRPS39uScoreHeritage:for IsGroup}
}\hfill{\scriptsize (attribute)}}\\
\textbf{\indent Returns:\ }
\texttt{list} 



 Returns as a list the following information for a group of order 19683 loaded
from the library \texttt{[ParentGroupID, ParentGroupOrder, Age]}. The \emph{Age} of a group is the position of the group among its siblings in the ordered list
of their standard PC codes. }

 

\subsection{\textcolor{Chapter }{FixGroupDescendants}}
\logpage{[ 2, 1, 4 ]}\nobreak
\hyperdef{L}{X81C0154E86A28E8B}{}
{\noindent\textcolor{FuncColor}{$\triangleright$\enspace\texttt{FixGroupDescendants({\mdseries\slshape arg})\index{FixGroupDescendants@\texttt{FixGroupDescendants}}
\label{FixGroupDescendants}
}\hfill{\scriptsize (function)}}\\


 

 }

 }

 }

 \def\bibname{References\logpage{[ "Bib", 0, 0 ]}
\hyperdef{L}{X7A6F98FD85F02BFE}{}
}

\bibliographystyle{alpha}
\bibliography{manualbib.xml}

\addcontentsline{toc}{chapter}{References}

\def\indexname{Index\logpage{[ "Ind", 0, 0 ]}
\hyperdef{L}{X83A0356F839C696F}{}
}

\cleardoublepage
\phantomsection
\addcontentsline{toc}{chapter}{Index}


\printindex

\immediate\write\pagenrlog{["Ind", 0, 0], \arabic{page},}
\newpage
\immediate\write\pagenrlog{["End"], \arabic{page}];}
\immediate\closeout\pagenrlog
\end{document}
